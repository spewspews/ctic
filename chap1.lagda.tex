\documentclass[12pt]{article}

\usepackage{fontspec}
\usepackage{unicode-math}
\usepackage{fancyvrb}

\DefineVerbatimEnvironment{code}{Verbatim}{baselinestretch=.8}

\setmainfont{Garamond Premier Pro}[Contextuals=AlternateOff]
\setmathfont{Garamond Math}[Scale=MatchUppercase]
\setmonofont{Monaspace Argon}[Scale=0.7]

\setlength{\parindent}{0em}

\begin{document}
\begin{code}
module chap1 where
open import Categories.Category using (Category)
open Category
open import Categories.Morphism
private
  variable
    a b : Category.Obj
\end{code}

\textsc{Exercise} 1.1.i.i: Show that a morphism can have at most one inverse isomorphism.
\vspace{10pt}

Given \(f:x→y\) and \(g,g':y→x\) with \(fg=1_y\), \(gf= 1_x\), \(fg' = 1_y\) and \(g'f=1_x\), then \(g=1_xg=g'fg=g'1_y=g'\)

\begin{code}
inverses-are-unique : ∀ (f : a ⇒ b) (g : b ⇒ a) (g' : b ⇒ a) → Iso f g → Iso f g' → g ≈ g'
\end{code}

\textsc{Exercise} 1.1.i.ii: Consider a morphism \(f: x → y\). Show  that if there exists a pair of morphisms \(g,h:y\rightrightarrows x\) so that \(gf=1_x\) and \(fh=1_y\), then \(g = h\) and \(f\) is an isomorphism.
\vspace{10pt}

Then \(g = g1_y = gfh = 1_xh = h\) so that \(fg = fh = 1_y\) and we already know that \(gf = 1_x\) hence \(f\) is an isomorphism.

\end{document}
